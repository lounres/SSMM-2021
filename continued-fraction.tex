\documentclass[12pt,a4paper]{article}
\usepackage{math-text}
\usepackage{todonotes}
\usepackage{multicol}
% \usepackage{float}

\title{Непрерывные дроби.}
\author{Сергей Константинович Ландо}

\begin{document}
    \maketitle

    Рассмотрим цепную дробь вида
    \begin{gather*}
        \dfrac{1}{
            1 - ct + \dfrac{ct^2}{
                1 - (c-2)t + \dfrac{(4c-3)t^2}{
                    1 - (c-6)t + \dfrac{(9c-18)t^2}{
                        \dots
                    }
                }
            }
        }\\
        = 1 + ct + c(c-1)t^2 + c(c-1)(c-2)t^3 + c(c^3 - 6c^2 + 13c - 7) + \dots = \sum_{k=0}^\infty L_n(c) t^n,
    \end{gather*}
    где каждый этаж $n$ выглядит как
    \[
        \dfrac{\dots}{
            1 - (c - n(n-1))t + \dfrac{(n^2 c - n^2(n^2 - 1))t^2}{
                \dots
            }
        }
    \]
    а $\deg(L_n) = n$.

    \begin{definition}
        \emph{Хордовая диаграмма.} Рассмотрим  окружность, в которой проведены хорды. При этом вершины хорд можно двигать как угодно с сохранением их порядка на окружности. Определим рекуррентно \emph{инвариант Чмутова-Варченко}:
        \begin{enumerate}
            \item Если диаграмма пуста, то сопоставим ей $1$.
            \item Если диаграмма состоит только из единственной хорды, то сопоставим ей $c$.
            \item Если диаграмма содержит ребро, пересекающееся ровно с одним другим, то сопоставим ей $(c-1) \cdot M$, где $M$ --- значение этой диаграммы, но без рассмотренного ребра.
            \item Если диаграмма состоит из двух непересекающихся наборов хорд, то значением равно произведению значений диаграмм для каждого из набора хорд.
            \item \todo[inline]{Добавить картинку.}
        \end{enumerate}
        \todo[inline]{Нарисовать.}
    \end{definition}

    \begin{theorem}
        Это определение корректно.
    \end{theorem}

    \begin{statement}[гипотеза, открыта]
        $L_n(c)$ есть инвариант Чмутова-Варченко для диаграммы из $n$ попарно пересекающихся хорд (главных диагоналей $2n$-угольника) для значения $c$.
    \end{statement}

    \begin{remark}
        Данное утверждение проверено для $n$ до $16$. Также доказано, что коэффициент при $c$ совпадает.
    \end{remark}

    \begin{definition}
        \[C(t) := \sum_{k=0}^\infty C_n t^n\]
        --- производящая функция чисел Каталана.
    \end{definition}

    Несложно видеть, что
    \begin{gather*}
        t C^2(t) = C(t) - 1\\
        C(t)(1 - t C(t)) = 1\\
        C(t)
        = \frac{1}{1 - t C(t)}
        = \dfrac{1}{
            1 - \dfrac{t}{
                1 - t C(t)
            }
        }
        = \dfrac{1}{
            1 - \dfrac{t}{
                1 - \dfrac{t}{
                    1 - \dfrac{t}{
                        \dots
                    }
                }
            }
        }
    \end{gather*}

    \begin{definition}
        \emph{Хроматический многочлен графа $G$} --- многочлен $\chi_G$, где $\chi_G(c)$ --- количество правильных раскрасок графа в $c$ цветов.
    \end{definition}

    \begin{example}
        $\chi_{K_n}(c) = A_c^n$.
    \end{example}

    \begin{exercise}
        \begin{gather*}
            \sum_{n=0}^\infty \chi_{K_n}(c) t^n = 1 + ct + c(c-1)t^2 + c(c-1)(c-2) t^3 + \dots\\
            = \dfrac{1}{
                1 - ct + \dfrac{ct^2}{
                    1 - (c-2) t + \dfrac{2(c-1) t^2}{
                        1 - (c-4) t + \dfrac{3(c-2) t^2}{
                            \dots
                        }
                    }
                }
            }
        \end{gather*}
    \end{exercise}

    \begin{theorem}[Чмутов, Ландо]
        Инвариант Чмутова---Варченко на хордовой диаграмме зависит только от графа пересечения хорд диаграммы.
    \end{theorem}

    \begin{exercise}
        Найти функции цепных дробей.
        \begin{multicols}{2}
            \begin{enumerate}
                \item
                    \[
                        \dfrac{1}{
                            1 - \dfrac{1^2 t^2}{
                                1 - \dfrac{2^2 t^2}{
                                    1 - \dfrac{3^2 t^2}{
                                        \dots
                                    }
                                }
                            }
                        }
                        = {?}
                    \]
                \item
                    \[
                        \dfrac{1}{
                            1 - \dfrac{1 \cdot 2 t^2}{
                                1 - \dfrac{2 \cdot 3 t^2}{
                                    1 - \dfrac{3 \cdot 4 t^2}{
                                        \dots
                                    }
                                }
                            }
                        }
                        = {?}
                    \]
            \end{enumerate}
        \end{multicols}
    \end{exercise}

    \begin{definition}
        \emph{Функция Ламберта} ---
        \[\Psi(t) := \sum_{n=1}^\infty \frac{n^{n-1}}{n!} t^n.\]
    \end{definition}

    \begin{lemma}
        \[\Psi(t) = t e^{\Psi(t)}.\]
    \end{lemma}
\end{document}