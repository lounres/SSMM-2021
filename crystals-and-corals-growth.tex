\documentclass[12pt,a4paper]{article}
\usepackage{math-text}
% \usepackage{todonotes}
% \usepackage{multicol}
% \usepackage{float}

\title{Как растут кристаллы и кораллы.}
\author{Виктор Алексеевич Клепцын}

\begin{document}
    \maketitle

    \begin{enumerate}
        \item Diffusion Limited Aggregation (DLA): T. A. Witten, L. M. Sander, 1981.
        \item Dielectric breakdown model (DBM): Niemeyer L., Pietronero L., Wiesmann H., 1984.
    \end{enumerate}

    Пусть даны слипшиеся $T$ частиц радиуса $\approx 1$ и на шаге $T+1$ случайно прилипает $T+1$-ая частица радиуса $1$. Тогда на момент $T$ радиус слипшейся фигуры --- $R = R(T)$. Понятно, что тогда
    \[\sqrt{T} \leqslant R(T) \leqslant T \qquad \text{т.е.} \qquad R \leqslant T \leqslant R^2.\]
    Оказывается, что $T \approx R(T)^\rho$, т.е. $R(T) \approx T^\beta$ ($\beta = 1/\rho \in [0.5; 1]$).

    \begin{theorem}[Harry Kesten]
        \[\beta \leqslant \frac{2}{3}.\]
    \end{theorem}

    
\end{document}