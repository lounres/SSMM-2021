\documentclass[12pt,a4paper]{article}
\usepackage{math-text}
% \usepackage{todonotes}
% \usepackage{multicol}
% \usepackage{float}

\title{Множество Мандельброта.}
\author{Владлен Анатольевич Тиморин}

\begin{document}
    \maketitle

    Множество Мандельброта появляется из динамических систем при рассмотрении орбит состояний, т.е. $\{f^n(x)\}_{n \in \NN}$.

    Будем рассматривать полиномиальную комплексную динамику: $f: \CC \to \CC$, $f \in \CC[x]$.

    \begin{lemma}
        Любой квадратный многочлен движением можно перевести к виду $z^2 + c$.
    \end{lemma}

    \begin{definition}
        \[Q_c(x) := x^2 + c\]
    \end{definition}

    \begin{definition}
        Пусть $P \in \CC[x]$.
        \begin{itemize}
            \item \emph{Заполненное множество Жюлиа} ---
                \[K(P) := \{z \in \CC \mid \lim_{n \to \infty} P^n(z) \neq \infty\}.\]
            \item \emph{Множество Жюлиа} ---
                \[J(P) := \partial K(P).\]
            \item \emph{Множество Фату} ---
                \[\CC P^1 \setminus J(P).\]
            \item \emph{Множество Мандельброта} ---
                \[\mathcal{M} := \{c \in \CC \mid K(Q_c) \text{ связно}\}.\]
        \end{itemize}
    \end{definition}

    \begin{theorem}
        Для всякой точки $c \in \CC$
        \[c \in \mathcal{M} \quad \Longleftrightarrow \quad \lim_{n \to \infty} Q_c^{\circ n}(0) \neq \infty,\]
        где $f^{\circ n} := \underbrace{f \circ \dots \circ f}_{n \text{ раз}}$.
    \end{theorem}

    \begin{proof}
        \begin{thlemma}
            Пусть $D$ --- открытая область, ограниченная простой замкнутой кривой. Тогда
            \begin{itemize}
                \item если $c \in D$, то $Q_c^{\circ -1}(D)$ --- одна область;
                \item если $c \notin D$, то $Q_c^{\circ -1}(D)$ --- две области.
            \end{itemize}
        \end{thlemma}
    
        \begin{proof}
            $0$ --- особая точка $Q_c$, поэтому $Q_c(0) = c$ --- особая точка $Q_c^{\circ -1}$. Поэтому, понятно, что $Q_c^{\circ -1}(D)$ получается из $D$ разрезом по радиусу из $c$, сжатием по углу относительно $c$ в два раза и дублированием с поворотом на $\pi$. Таким образом если $c \in D$, то дубликаты склеятся в одну область, а иначе нет.
        \end{proof}

        Для всякого $c \in \CC$, что $\lim_{n \to \infty} Q_c^{\circ n}(0) \neq \infty$, возьмём очень большой диск $D_0 = B_r(0)$, где $r \gg c$. Тогда всякая точка вне $D_0$ убегает на бесконечность. Следовательно $K(Q_c) \subseteq D_0$, и по той же причине $Q_c^{\circ -1}(D_0) \subseteq D_0$. Поскольку $c \in D_0$, то $D_n := Q_c^{\circ -n}(D_0)$ содержит $c$ и $K(Q_c)$, а значит $D_n$ --- ``диск''. Таким образом
        \[K(Q_c) = \bigcap_{n=0}^\infty D_n,\]
        так как $\lim_{n \to \infty} Q_c^{\circ n}(z) \neq \infty$ тогда и только тогда, когда $z$ лежит во всех $D_n$, так как $Q_c^{\circ n}(z) \in D_0$. Следовательно $K(Q_c)$ как пересечение ``дисков'' является связным.
        
        Аналогично, если $\lim_{n \to \infty} Q_c^{\circ n}(0) = \infty$, то $c$ не будет содержаться в $D_n$ начиная с некоторого $n$, следовательно $D_n$ с некоторого момента перестанут быть связными, а с ними и $K(Q_c)$. Таким образом получаем равносильность.
    \end{proof}
\end{document}