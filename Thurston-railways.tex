\documentclass[12pt,a4paper]{article}
\usepackage{math-text}
\usepackage{todonotes}
% \usepackage{multicol}
% \usepackage{float}

% \DeclareMathOperator{\Iso}{Iso}
% \DeclareMathOperator{\PSL}{PSL}
\newcommand{\Iso}{\mathrm{Iso}}
\newcommand{\PSL}{\mathrm{PSL}}
\renewcommand{\Re}{\mathrm{Re}}
\renewcommand{\Im}{\mathrm{Im}}
\newcommand{\Diff}{\mathrm{Diff}}
\newcommand{\id}{\mathrm{id}}
\newcommand{\Mod}{\mathrm{Mod}}
\newcommand{\Teich}{\mathrm{Teich}}

\title{Слоения, железные дороги Терстона и гиперболическая геометрия на поверхностях.}
\author{Гаянэ Юрьевна Панина}

\begin{document}
    \maketitle

    \begin{definition}
        $S_g$ --- поверхность рода $g$, т.е. связная ориентируемая компактная поверхность. Говоря иначе, это сфера $g$ ручками.
    \end{definition}

    \begin{lemma}
        $\chi(S_g) = 2 - 2g$ --- характеристика Эйлера --- есть полный инвариант.
    \end{lemma}

    \begin{definition}
        Поверхность рода $g$ с $n$ дырками --- связная компактная ориентируемая поверхность с краем(-ями). Говоря иначе, сфера с $g$ ручками и $n$ дырками.
    \end{definition}

    \begin{lemma}
        $(\chi; n)$ (или же, что равносильно, $(g; n)$) есть полный инвариант.
    \end{lemma}

    \begin{definition}
        \emph{Замкнутая кривая $\gamma$} --- непрерывное отображение $\gamma: S^1 \to S_g$, что
        \begin{itemize}
            \item $\gamma$ без самопересечений,
            \item $\gamma$ гладка,
            \item $\gamma$ с точностью до изотопии (гомотопии).
        \end{itemize} без самопересечений.
    \end{definition}

    \begin{definition}
        \emph{Гомеоморфизм} --- отображение между топологическими пространствами
        \[\varphi: X \to Y,\]
        что
        \begin{itemize}
            \item $\varphi$ --- биекция,
            \item $\varphi$ --- непрерывна,
            \item $\varphi^{-1}$ --- непрерывна.
        \end{itemize}
    \end{definition}

    \begin{definition}
        Диффеоморфизм --- гладкий гомеоморфизм.
    \end{definition}

    \begin{remark}
        Для рассматриваемых пространств верно, что гомеоморфные пространства диффеоморфны.
    \end{remark}

    \begin{example}\ 
        \begin{enumerate}
            \item Тождественное отображение на $X$ --- диффеоморфизм.
            \item К диффеоморфизмам можно применять изотопии.
        \end{enumerate}
    \end{example}

    \begin{problem}
        Любую неразбивающую кривую $\gamma$ на $X$ (т.е. $X \setminus \gamma$ линейно связно) можно перевести в любую другую неразбивающую.
    \end{problem}

    \begin{definition}
        \emph{Изометрия} --- гомеоморфизм, сохраняющий метрику (или, что равносильно, длины кривых). 
    \end{definition}

    \begin{problem}\ 
        \begin{enumerate}
            \item \textbf{Первый тор.} Рассмотрим квадрат $[0; 1]^2$. Склеим его обычным способом в тор. Получим \emph{тор, снабжённый плоской метрикой}, т.е. у каждой точки есть окрестность изометричная диску.
            \item \textbf{Второй тор.} Сделаем то же самое, но для параллелограмма натянутого на $(1; 0)$ и $(1; 1)$.
            \item \textbf{Третий тор.} То же самое, но для параллелограмма натянутого на $(1; 0)$ и $(0.5; 1)$.
        \end{enumerate}
        Изометричны ли торы?
    \end{problem}

    \begin{remark}
        Нельзя склеить поверхность рода $g$ из плоскости. Действительно, если, например, взять обычную развёртку $\prod_{i=1}^n (a_i b_i a_i^{-1} b_i^{-1})$. Все вершины будут склеены в одну и сумма углов банально не сойдётся (будет очень большой).

        С другой стороны рассмотрим модель плоскости Лобачевского через ортогональные к окружности дуги внутри окружности. Если возьмём правильный $4g$-угольник с центром в центре нашей плоскости очень малого размера, то её сумма углов будет больше $2\pi$. Если же взять $4g$-угольник, вершины которого бесконечно удалены (лежат на границе нашей плоскости), то сумма углов будет равна $0 < 2\pi$. Значит где-то ``посередине'' будет $4g$-угольник с суммой углов $2\pi$. В таком случае склеивая такой многоугольник таким же образом, мы получаем плоскую метрику. Она называется \emph{гиперболической метрикой с постоянной кривизной $-1$}.
    \end{remark}

    \begin{definition}
        Модель Пуанкаре плоскости Лобачевского --- $\HH := \{z \in \CC \mid \Im(z) > 0\}$, где прямые --- окружности, перпендикулярные $\Im(z) = 0$, а метрика порождается формулой
        \[ds = \frac{\sqrt{dx^2 + dy^2}}{y}.\]

        Множество изометрий модели Пуанкаре $\Iso^+(\HH)$ --- дробно-рациональные функции с вещественными коэффициентами и положительным определителем, т.е. $\PSL(2, \RR)$.
    \end{definition}

    \begin{remark}
        По теореме Брауера у всякой изометрии $\HH$ есть неподвижная в замыкании $\HH$. Если она лежит в $\HH$, то такая изометрия равносильно повороту относительно центра в представлении плоскости Лобачевского в качестве круга, где неподвижная точка --- центр. Если же она лежит на границе, то это равносильно параллельному переносу на $a \in \RR$ в модели плоскости Лобачевского, где неподвижная точка --- бесконечно-удалённая точка. 
    \end{remark}

    \begin{theorem}
        Рассмотрим на $S_g$ замкнутую простую существенную (т.е. нестягиваемую) кривую $c$. Пусть на $S_g$ есть гиперболическая метрика $\tau$. Тогда
        \begin{enumerate}
            \item Существует и, если $g \neq 1$, единственна кривая $c' \approx c$ минимальной длины.
            \item $c'$ --- геодезическая.
        \end{enumerate}
    \end{theorem}

    \begin{proof}
        В случае тора мы имеем, что он склеивается из квадрата $[0; 1]^2$. Таким образом можно рассмотреть поднятие нашей кривой $c$ на $\RR^2$. Рассмотрим её поднятие между двумя какими-нибудь эквивалентными (относительно $\ZZ^2$) соседними по $c$ точками и натянем между ними. Получим $c'$ минимальной длины, гомотопную $c$. Но также её можно подвигать параллельно, что означает неединственность минимальной кривой.

        Точно также можно замостить $\HH$.
        \todo[inline]{Дописать.}
    \end{proof}

    \begin{problem}
        Пусть на поверхности $m$ выбрана простая кривая $\gamma$. Определим метрику, в которой длина кривой $c$ есть
        \[l_\gamma(c) := \min_{c' \approx c} |\gamma \cap c|.\]

        Доказать, что $c$ реализует рассматриваемое минимальное число пересечений тогда и только тогда, когда $\gamma$ и $c$ не образуют двуугольников (дисков, полвоина границы которых есть часть $c$, а другая половина --- часть $\gamma$).
    \end{problem}

    \begin{definition}\ 
        \begin{itemize}
            \item $\Diff(S_g)$ --- группа диффеоморфизмов $S_g$.
            \item $\Diff_0(S_g)$ --- группа диффеоморфизмов, изотопных $\id$.
            \item $\Mod(S_g) := \Diff / \Diff_0$.
        \end{itemize}
    \end{definition}

    \begin{definition}
        Пространство Тейхмюллера ---
        \[\Teich(S_g) := \{\text{гиперболическая структура на $S_g$}\} / \Diff_0.\]
    \end{definition}

    \begin{theorem}
        \[\Teich(S_g) = \RR^{6g-6}.\]
    \end{theorem}

    \begin{proof}
        Разрежем $S_g$ по геодезическим кривым на ``штаны'' (сферы с 3 дырками). Итого ``штанов'' у нас $2g-2$, а разрезов $3g-3$. Далее у каждых штанов каждую пару краёв соединим минимальной кривой и разрежем по ней. Получим $6g-6$ шестиугольников с прямыми углами. При этом пусть в одних штанах длины разрезов --- $l_1$, $l_2$ и $l_3$.
        
        \begin{lemma}
            Шестиугольник с прямыми углами со сторонами $l_1$, $x$, $l_2$, $y$, $l_3$, $z$ по циклу существуют и единственны.
        \end{lemma}

        Таким образом каждые штаны распадаются на двое одинаковых шестиугольника, а значит определяются по длинам трёх краёв. Таким образом множество штанов определяется по длинам разрезов, что даёт $3g-3$ параметра (по $3g-3$ разрезам). Далее мы можем склеить эти разрезы с любой закруткой $\in (-\infty; \infty)$. Таким образом
        \[\Teich(S_g) \simeq (\RR_{>0} \times \RR)^{3g-3} \simeq \RR^{6g-6}.\]
    \end{proof}
\end{document}