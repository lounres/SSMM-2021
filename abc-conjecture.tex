\documentclass[12pt,a4paper]{article}
\usepackage{math-text}
% \usepackage{todonotes}
% \usepackage{multicol}
% \usepackage{float}

\newcommand{\rad}{\mathrm{rad}}

\title{ABC-гипотеза и ее следствия.}
\author{Дмитрий Олегович Орлов}

\begin{document}
    \maketitle

    \begin{statement}[(бинарная) гипотеза Гольдбаха, открыта]
        Всякое чётное число, начиная с $4$, можно представить в виде суммы 2 простых чисел.
    \end{statement}

    \begin{statement}[(тернарная) гипотеза Гольдбаха, доказана]
        Всякое нечётное число, начиная с $7$, можно представить в виде суммы 3 простых чисел.
    \end{statement}
    
    \begin{statement}[гипотеза близнецов, открыта]
        Существует бесконечно много пар простых чисел вида $(p; p+2)$.
    \end{statement}
    
    \begin{statement}[открыта]
        Для всякого $n$ есть простое $p \in [n^2; (n+1)^2]$.
    \end{statement}
    
    \begin{statement}[открыта]
        Существует бесконечно много простых чисел вида $n^2 + 1$.
    \end{statement}

    \begin{definition}
        Для всякого натурального числа $n = \prod_{i=1}^k p_i^{\alpha_i}$ назовём его \emph{радикалом} число
        \[\rad(n) := \prod_{i=1}^k p_i.\]
    \end{definition}
    
    \begin{statement}[$abc$-гипотеза, открыта]
        Для всякого $\varepsilon > 0$ существует постоянная $K(\varepsilon)$, что для всякой тройки $a$, $b$, $c$ попарно взаимно натуральных чисел, что $a + b = c$, выполняется неравенство
        \[c \leqslant K(\varepsilon) \cdot \rad(abc)^{1 + \varepsilon}.\]
    \end{statement}

    \begin{example}
        Есть бесконечно много троек, где $c \geqslant \rad(abc)$. Действительно, если $a = 1$, $b = 3^{2^k} - 1$, $c = 3^{2^k}$, то по лемме об уточнении показателя $\nu_2(b) = k+2$, а значит
        \[\rad(abc) \leqslant \frac{3b}{2^{k+1}} \ll c.\]
    \end{example}

    \begin{example}
        Пусть
        \[\rho(a, b, c) := \frac{\ln(c)}{\ln(\rad(abc))}\]
        Самые известные ``плохие'' примеры:
        \begin{enumerate}
            \item $a = 2$, $b = 3^{10} \cdot 109$, $c = 23^5$,
                \[\rho \approx 1{,}62991.\]
            \item $a = 11^2$, $b = 3^2 \cdot 5^6 \cdot 7^3$, $c = 2^{21} \cdot 23$,
                \[\rho \approx 1{,}62599.\]
        \end{enumerate}
    \end{example}
    
    \begin{statement}[открыта]
        Для всякой тройки $a$, $b$, $c$ попарно взаимно натуральных чисел, что $a + b = c$, выполняется неравенство
        \[c \leqslant \rad(abc)^2.\]
    \end{statement}
    
    \begin{corollary}[почти великая теорема Ферма]
        Уравнение
        \[x^2 + y^2 = z^2\]
        не имеет нетривиальных решений при $n > 6$.
    \end{corollary}

    \begin{proof}
        Пусть $a = x^n$, $b = y^n$, $c = z^n$. Тогда
        \[z^n = c \leqslant \rad(abc)^2 \leqslant (xyz)^2 \leqslant z^6\]
        --- противоречие.
    \end{proof}
    
    \begin{statement}[гипотеза Морделла, доказана]
        Количество рациональных точек на алгебраической кривой рода $\geqslant 2$ конечно.
    \end{statement}
\end{document}