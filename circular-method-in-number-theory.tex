\documentclass[12pt,a4paper]{article}
\usepackage{math-text}
\usepackage{todonotes}
% \usepackage{multicol}
% \usepackage{float}

\newcommand{\GCD}{\mathrm{GCD}}
\newcommand{\const}{\mathrm{const}}

\title{Основы кругового метода в теории чисел.}
\author{Максим Александрович Королёв}

\begin{document}
    \maketitle

    Мы будем задаваться вопросом вида когда и сколькими способами некоторое целое число $N$ раскладывается в сумму
    \[N = a_1 + \dots + a_k, \qquad a_i \in A.\]

    \begin{example}
        Теорема Лагранжа о разложимости числа в сумму 4 квадратов есть частный случай для $k = 4$, $A = \{a^2\}_{a \in \ZZ}$.
    \end{example}

    \begin{statement}[проблема Варинга, доказана]
        Для всякого $n > 1$ есть константа $k = k(n)$, что всякое натуральное $N$ представимо в виде
        \[N = x_1^n + \dots + x_k^n,\]
        где $x_i \geqslant 0$.
    \end{statement}

    \begin{definition}
        Для всякого целого $m$ определим
        \[\delta(m) := \int_0^1 e^{2\pi i m \alpha} d\alpha.\]
    \end{definition}

    \begin{remark}
        $\delta(0) = 1$. Если $m \neq 0$, то
        \[\delta(m) = \left.\frac{e^{2\pi i \alpha m}}{2 \pi i m}\right|_0^1 = 0.\]
        Следовательно
        \[\delta(m) = [m = 0].\]
    \end{remark}

    Пусть $I(N)$ --- число решений уравнения
    \[N = x_1^3 + \dots + x_10^3.\]
    Рассмотрим случайный вектор $x = (x_1, \dots, x_10)$ и обозначим
    \[m = x_1^3 + \dots x_10^3 - N.\]
    Следовательно $\delta(m) = 1$ тогда и только тогда, когда $x$ --- корень. Понятно, что $x_i \leqslant \sqrt[3]{N} =: P$. Следовательно
    \begin{align*}
        I(N)
        &= \sum_{0 \leqslant x_1, \dots, x_10 \leqslant P} \delta(x_1^3 + \dots x_10^3 - N)\\
        &= \sum_{\vec{x}} \int_0^1 e^{2\pi i \alpha (x_1^3 + \dots + x_10^3 - N)} d\alpha\\
        &= \int_0^1 \prod_{j=1}^{10} \sum_{0 \leqslant x_j \leqslant P} e^{2\pi i \alpha x_j^3} \cdot e^{-2\pi i \alpha N} d\alpha\\
        &= \int_0^1 S_3^{10}(\alpha) e^{-2\pi i \alpha N} d\alpha.
    \end{align*}

    $J(N)$ --- частное решение уравнения Лагранжа.
    \[J(N) = \int_0^1 S_2^4(\alpha) e^{-2\pi \alpha N} d\alpha.\]
    Имеем, что $J(N) > 0$ для всякого $N \geqslant 1$. Также получаем, что
    \[J(N) = \int_0^1 W^3(\alpha) e^{-2\pi \alpha N} d\alpha,\]
    где
    \[W(\alpha) = \sum_{p \leqslant N} e^{2\pi \alpha p}.\]

    \begin{theorem}
        Пусть есть $\tau > 1$. Тогда для всякого $\alpha \in [0; 1]$ существуют целые $0 \leqslant a \leqslant q$, что $\GCD(a, q) = 1$, $1 \leqslant q \leqslant \tau$, что
        \[|\alpha - \frac{p}{q}| \leqslant \frac{1}{q \tau}.\]
    \end{theorem}

    \begin{proof}
        Рассмотрим $\alpha_j := \{j\alpha\}$. Возьмём $n = [\tau] + 1$. Тогда
        \[0 = \alpha_0 \leqslant \alpha_1 \leqslant \dots \leqslant \alpha_n \leqslant \alpha_{n+1} = 1\]
        --- разбиение $[0; 1]$. Тогда существуют $k$ и $m$, что
        \[0 \leqslant \alpha_k - \alpha_m \leqslant \frac{1}{n+1}.\]

        Пусть $1 \leqslant m \leqslant k \leqslant n$. Тогда
        \begin{gather*}
            0 \leqslant k \alpha - [k \alpha] - m \alpha + [m \alpha] \leqslant \frac{1}{n+1}\\
            0 \leqslant \alpha(k-m) - ([k \alpha] - [m \alpha]) \leqslant \frac{1}{n+1}\\
            0 \leqslant \alpha - \frac{[k \alpha] - [m \alpha]}{k-m} \leqslant \frac{1}{(n+1)(k-m)}
        \end{gather*}
        Пусть
        \[\frac{a}{q} := \frac{[k \alpha] - [m \alpha]}{k-m}.\]
        Тогда
        \[0 \leqslant \alpha - \frac{a}{q} \leqslant \frac{1}{(n+1)(k-m)} \leqslant \frac{1}{\tau q}.\]
    \end{proof}

    \begin{remark}
        Такая дробь $\frac{a}{q}$ называется \emph{рациональным приближением (к) $\alpha$ порядка $\tau$}.
    \end{remark}

    \begin{remark}
        Если $\frac{a}{q}$ --- рациональное приближение порядка $\tau$, то $\alpha = \frac{a}{q} + \frac{\theta}{q \tau}$, где $|\theta| \leqslant 1$.
    \end{remark}

    \begin{remark}
        Пусть
        \[E(a, q) := (\frac{a}{q} - \frac{1}{q \tau}; \frac{a}{q} + \frac{1}{q \tau}).\]
        Тогда $\frac{a}{q}$ будет приближением порядка $\tau$ тогда и только тогда, когда $\alpha \in E(a, q)$.
    \end{remark}

    Теперь пусть $\tau = 6 P^2 = 6N$. Берём $Q \in (1; \frac{\tau}{2})$ (конкретное значение будет позже). Отрезок $[0; 1]$ покрыт объединением
    \[\bigcup_{1 \leqslant q \leqslant \tau} \bigcup_{\substack{a = 0\\ (a, q) = 1}}^q E(a, q).\]
    Пусть $E_1$ --- объединение интервалов $E(a, q)$ с $q \leqslant Q$, а
    \[E_2 := (\frac{-1}{\tau}; 1 - \frac{1}{\tau}] \setminus E_1\]
    --- всё остальное.

    \begin{theorem}
        $E_1$ состоит из непересекающихся интервалов.
    \end{theorem}

    \begin{proof}
        Предположим противное. Пусть есть точки $\frac{a_1}{q_1}$ и $\frac{a_2}{q_2}$, отрезки которых перекрываются. Следовательно
        \begin{gather*}
            0 < \frac{a_2}{q_2} - \frac{a_1}{q_1} < \frac{1}{q_1 \tau} + \frac{1}{q_2 \tau}\\
            0 < \frac{a_2 q_1 - a_1 q_2}{q_1 q_2} < \frac{q_1 + q_2}{q_1 q_2 \tau}\\
            \tau < q_1 + q_1 \leqslant 2Q
        \end{gather*}
    \end{proof}

    \[I(N) = \int_{E_1} + \int_{E_2} = I_1(N) + I_2(N)\]
    --- \emph{интегралы по большим и малым дугам}.

    Тригонометрические суммы:
    \[S_1(\alpha) := \sum_{x=0}^P e^{2 \pi i \alpha x}.\]
    Если $\alpha$ целое, то $S_1 = P+1$, иначе
    \[S_1(\alpha) = \frac{e^{2 \pi i \alpha (P+1)} - 1}{e^{2 \pi i \alpha} - 1}.\]
    Таким образом
    \[|S_1(\alpha)| \leqslant \frac{2}{|e^{\pi i \alpha}(e^{\pi i \alpha} - e^{-\pi i \alpha})|} = \frac{1}{\sin(\pi \alpha)} = \frac{1}{\sin(\pi \|\alpha\|)} \leqslant \frac{1}{2 \|\alpha\|},\]
    где
    \[\|\alpha\| := \min_{n \in \ZZ}(|\alpha - n|)\]
    --- расстояние от $\alpha$ до ближайшего целого. Таким образом
    \[|S_1(\alpha)| \leqslant \min\left(P+1, \frac{1}{2\|\alpha\|}\right)\]

    Теперь давайте оценим $I_2(N)$. Квадратичная сумма:
    \[S_2(\alpha, \beta) = \sum_{0 \leqslant x \leqslant P} e^{2\pi i (\alpha x^2 + \beta x)}.\]
    Тогда
    \[
        S_2^2
        = \overline{S}_2 \cdot S_2
        = \sum_{0 \leqslant x \leqslant P} e^{-2\pi i (\alpha x^2 + \beta x)} \sum_{0 \leqslant y \leqslant P} e^{2\pi i (\alpha y^2 + \beta y)}
        = \sum_{0 \leqslant x, y \leqslant P} e^{2\pi i (\alpha (y^2 - x^2) + \beta (y - x))}
    \]
    Фиксируя $x$ и делая замену $h := y - x \in [-x; P - x]$, получаем, что
    \[
        S_2^2
        = \sum_{0 \leqslant x \leqslant P} \sum_{-x \leqslant h \leqslant P-x} e^{2\pi i (\alpha h^2 + \beta h)} \cdot e^{2\pi i \cdot 2\alpha hx}
    \]
    Пусть $x_1 := \min(0, -h)$, $y_1 := \min(P, P-h)$. Тогда
    \[
        S_2^2
        = \sum_{|h| \leqslant P} e^{2\pi i (\alpha h^2 + \beta h)} \sum_{x_1 \leqslant x \leqslant y_1} e^{2\pi i \cdot 2\alpha hx}
        \leqslant \sum_{|h| \leqslant P} \min\left(P+1, \frac{1}{2 \|2\alpha h\|}\right)
        \leqslant P+1 + 2 \sum_{1 \leqslant h \leqslant P} \min\left(P+1, \frac{1}{2 \|2\alpha h\|}\right)
        \leqslant P+1 + 2 \sum_{1 \leqslant h \leqslant 2P} \min\left(P+1, \frac{1}{\|\alpha h\|}\right)
    \]

    \begin{definition}
        Знак Виноградова --- если $|A| \leqslant c \cdot B$, то
        \[A \ll B.\]
        Если $c = c(\alpha, \beta, \dots)$, то
        \[A \ll_{\alpha, \beta, \dots} B.\]
    \end{definition}

    \begin{lemma}[неравенство Гёльдера]
        Пусть даны $a_m \geqslant 0$. Тогда
        \[\left(\sum_{m=1}^M a_m\right)^k \leqslant M^{k-1} \sum_{m=1}^M a_m^k.\]
    \end{lemma}

    \begin{corollary}
        Если $M, k = \const$, то
        \[\left(\sum_{m=1}^M a_m\right)^k \ll \sum_{m=1}^M a_m^k.\]
    \end{corollary}

    Кубическая сумма:
    \[S_3(\alpha) = \sum_{0 \leqslant x \leqslant P} e^{2\pi i \alpha x^3}.\]
    Тогда
    \[
        |S_3|^2
        = \sum_{0 \leqslant x \leqslant P} \sum_{-x \leqslant h \leqslant P-x} e^{2\pi i \alpha ((x+h)^3 - x^3)}
        = \sum_{|h| \leqslant P} e^{2\pi i \alpha h^3} \sum_{x_1 \leqslant x \leqslant y_1} e^{2\pi i \cdot 3\alpha h(x^2 + xh)}
        \leqslant \sum_{|h| \leqslant P} \left|\sum_{x_1 \leqslant x \leqslant y_1} e^{2\pi i \cdot 3\alpha h(x^2 + xh)} \right|
        \leqslant P+1 + \sum_{1 \leqslant |h| \leqslant P} \left|\sum_{x_1 \leqslant x \leqslant y_1} e^{2\pi i \cdot 3\alpha h(x^2 + xh)} \right|
    \]
    Таким образом
    \begin{align*}
        |S_3|^4
        &\ll P^2 + \left(\sum_{|h|} \left|\sum_{x} \dots\right|\right)^2\\
        &\ll P^2 + P \sum_{1 \leqslant |h| \leqslant P} \left|\sum_{x} e^{2\pi i 3\alpha h(x^3 + xh)}\right|\\
        &\ll P^2 + P \sum_{1 \leqslant |h_1| \leqslant P} \left(P + \sum_{|h_2|} \min\left(P, \frac{1}{\|3 \alpha h_1 h_2\|}\right)\right)\\
        &\ll P^3 + P \sum_{1 \leqslant |h_1| \leqslant P} \sum_{1 \leqslant h_2 \leqslant 2P} \min\left(P, \frac{1}{\|3 \alpha h_1 h_2\|}\right)\\
        &\ll P^3 + P \sum_{1 \leqslant h_1 \leqslant P} \sum_{1 \leqslant h_2 \leqslant 2P} \min\left(P, \frac{1}{\|\alpha h_1 h_2\|}\right)\\
        &\ll P^3 + P \sum_{1 \leqslant n \leqslant 6P^2} \tau(n) \min\left(P, \frac{1}{\|\alpha n\|}\right)\\
        &\ll P^3 + P^{1+\varepsilon} \sum_{1 \leqslant n \leqslant 6P^2} \min\left(P, \frac{1}{\|\alpha n\|}\right)
    \end{align*}
    \todo[inline]{Разобраться с переходом к $\tau$.}

    \begin{theorem}
        Пусть $m \in \ZZ$, $x \geqslant 1$, $\alpha = \frac{a}{q} + \frac{\theta}{q^2}$, $|\theta| \leqslant 1$, $(a, q) = 1$, $q \geqslant 6$,
        \[W = \sum_{m - \frac{q}{2} < n \leqslant m + \frac{q}{2}} \min\left(x, \frac{1}{\|\alpha n\|}\right).\]
        Тогда $W \leqslant 4X + 2q \ln(q)$.
    \end{theorem}

    \begin{proof}
        Пусть $n = m + x$, $\frac{q}{2} < x \leqslant \frac{q}{2}$. Тогда
        \[\alpha n = \frac{ax+b+\theta}{q} + \frac{\theta x}{q^2}, \]
        \todo[inline]{Дописать.}
    \end{proof}

    Так мы получаем
    \begin{align*}
        |S_3|^4
        &\ll P^3 + P^{1+\varepsilon} \left(\frac{6P^2}{q} + 1\right) (4P + 2q \ln(q))\\
        &\ll P^3 + P^{1+\varepsilon} \left(\frac{P^3}{q} + P^2  \ln(q) + q \ln(q)\right)\\
        &\ll P^{1+\varepsilon} \left(\frac{P^3}{4} + P^2 \ln(q) + q\ln(q)\right)\\
        &= P^{4+\varepsilon} \left(\frac{1}{4} + \frac{\ln(q)}{P} + \frac{q \ln(q)}{P^2}\right).
    \end{align*}
\end{document}