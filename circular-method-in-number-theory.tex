\documentclass[12pt,a4paper]{article}
\usepackage{math-text}
\usepackage{todonotes}
% \usepackage{multicol}
% \usepackage{float}

\newcommand{\GCD}{\mathrm{GCD}}
\newcommand{\const}{\mathrm{const}}

\title{Основы кругового метода в теории чисел.}
\author{Максим Александрович Королёв}

\begin{document}
    \maketitle

    Мы будем задаваться вопросом вида когда и сколькими способами некоторое целое число $N$ раскладывается в сумму
    \[N = a_1 + \dots + a_k, \qquad a_i \in A,\]
    где мы фиксируем интересующее нас множество $A$.

    \begin{example}
        Теорема Лагранжа о разложимости числа в сумму 4 квадратов есть частный случай для $k = 4$, $A = \{a^2\}_{a \in \ZZ}$.
    \end{example}

    \begin{statement}[проблема Варинга, доказана]
        Для всякого $n > 1$ есть константа $k = k(n)$, что всякое натуральное $N$ представимо в виде
        \[N = x_1^n + \dots + x_k^n,\]
        где $x_i \geqslant 0$.
    \end{statement}

    Чтобы понять, как работает круговой метод, попробуем для примера доказать, что всякое целое число представимо в виде суммы 10 кубов.

    \begin{definition}
        Для всякого целого $m$ определим
        \[\delta(m) := \int_0^1 e^{2\pi i m \alpha} d\alpha.\]
    \end{definition}

    \begin{remark}
        $\delta(0) = 1$. Если $m \neq 0$, то
        \[\delta(m) = \left.\frac{e^{2\pi i \alpha m}}{2 \pi i m}\right|_0^1 = 0.\]
        Следовательно
        \[\delta(m) = [m = 0],\]
        где $[*]$ --- скобка Айверсона.
    \end{remark}

    Пусть $I(N)$ --- число решений уравнения
    \[N = x_1^3 + \dots + x_{10}^3.\]
    Рассмотрим случайный вектор $x = (x_1, \dots, x_{10})$ и обозначим
    \[m = x_1^3 + \dots x_{10}^3 - N.\]
    Следовательно $\delta(m) = 1$ тогда и только тогда, когда $x$ --- корень. Понятно, что $x_i \leqslant \sqrt[3]{N} =: P$. Следовательно
    \begin{align*}
        I(N)
        &= \sum_{0 \leqslant x_1, \dots, x_{10} \leqslant P} \delta(x_1^3 + \dots x_{10}^3 - N)\\
        &= \sum_{\vec{x}} \int_0^1 e^{2\pi i \alpha (x_1^3 + \dots + x_{10}^3 - N)} d\alpha\\
        &= \int_0^1 \sum_{0 \leqslant x_1, \dots, x_{10} \leqslant P} \prod_{j=1}^{10} e^{2\pi i \alpha x_j^3} \cdot e^{-2\pi i \alpha N} d\alpha\\
        &= \int_0^1 \prod_{j=1}^{10} \left(\sum_{0 \leqslant x \leqslant P} e^{2\pi i \alpha x^3}\right) \cdot e^{-2\pi i \alpha N} d\alpha\\
        &= \int_0^1 S_3(\alpha)^{10} e^{-2\pi i \alpha N} d\alpha,
    \end{align*}
    где
    \[S_n(\alpha) := \sum_{x=0}^P e^{2\pi i \alpha x^n}.\]

    Теорема же Лагранжа на таком языке будет записана как
    \[J(N) := \int_0^1 S_2^4(\alpha) e^{-2\pi \alpha N} d\alpha > 0.\]
    А тернарная теорема Гольдбаха на таком языке имеет вид
    \[J(N) = \int_0^1 W(\alpha)^3 e^{-2\pi \alpha N} d\alpha,\]
    где
    \[W(\alpha) = \sum_{\substack{p \leqslant N\\ p \in \PP}} e^{2\pi \alpha p}.\]

    \begin{theorem}
        Пусть есть $\tau > 1$. Тогда для всякого $\alpha \in [0; 1]$ существуют целые $0 \leqslant a \leqslant q$, что $\GCD(a, q) = 1$, $1 \leqslant q \leqslant \tau$, что
        \[\left|\alpha - \frac{a}{q}\right|  \leqslant \frac{1}{q \tau}.\]
    \end{theorem}

    \begin{proof}
        Возьмём $n = [\tau] + 1$. Рассмотрим $\alpha_j := \{j\alpha\}$ для $j \in [1; n]$, $\alpha_0 := 0$, а $\alpha_{n+1} := 1$. Тогда
        \[0 = \alpha_{i_0} \leqslant \alpha_{i_1} \leqslant \dots \leqslant \alpha_{i_n} \leqslant \alpha_{i_{n+1}} = 1\]
        --- разбиение $[0; 1]$, где $(i_k)_{k=0}^{n}$ --- некоторая перестановка $(0; 1; \dots; n+1)$. Понятно, что $i_0 = 0$, $i_{n+1} = n+1$. Тогда по принципу Дирихле существуют $k$ и $m$, что
        \[0 \leqslant \alpha_{k} - \alpha_{m} \leqslant \frac{1}{n+1}.\]

        Пусть $1 \leqslant m \leqslant k \leqslant n$. Тогда
        \begin{gather*}
            0 \leqslant k \alpha - [k \alpha] - m \alpha + [m \alpha] \leqslant \frac{1}{n+1}\\
            0 \leqslant \alpha(k-m) - ([k \alpha] - [m \alpha]) \leqslant \frac{1}{n+1}\\
            0 \leqslant \alpha - \frac{[k \alpha] - [m \alpha]}{k-m} \leqslant \frac{1}{(n+1)(k-m)}
        \end{gather*}
        Определим $a$ и $q$ так, что
        \[\frac{a}{q} := \frac{[k \alpha] - [m \alpha]}{k-m}.\]
        Тогда $q \mid k-m$, и, следовательно, $q \leqslant k-m$.
        \[0 \leqslant \alpha - \frac{a}{q} \leqslant \frac{1}{(n+1)(k-m)} \leqslant \frac{1}{\tau q}.\]

        Случай $1 \leqslant k \leqslant m \leqslant n$ рассматривается в точности также, только вместо $k-m$ надо рассматривать $m-k$.

        Случай $k = 0$ может выполниться, только если $\alpha_m = 0$ (а в таком случае $\alpha m \in \ZZ$, следовательно достаточно взять $a := \alpha m$, $q := m$). Аналогично (но уже строго) не может выполниться случай $m = n+1$.

        Пусть $m = 0$. Тогда $k \in [1; n]$, а значит
        \begin{gather*}
            0 \leqslant k \alpha - [k \alpha] \leqslant \frac{1}{n+1}\\
            0 \leqslant \alpha - \frac{[k \alpha]}{k} \leqslant \frac{1}{k(n+1)}
        \end{gather*}
        Определим $a$ и $q$ так, что
        \[\frac{a}{q} := \frac{[k \alpha]}{k}.\]
        Тогда $q \mid k$, и, следовательно, $q \leqslant k$.
        \[0 \leqslant \alpha - \frac{a}{q} \leqslant \frac{1}{k(n+1)} \leqslant \frac{1}{q \tau}.\]

        Пусть $k = n+1$. Тогда $m \in [1; n]$, а значит
        \begin{gather*}
            0 \leqslant 1 - m \alpha + [m \alpha] \leqslant \frac{1}{n+1}\\
            0 \leqslant \frac{1 - [m \alpha]}{m} - \alpha \leqslant \frac{1}{(n+1)m}\\
        \end{gather*}
        Определим $a$ и $q$ так, что
        \[\frac{a}{q} := \frac{1 - [m \alpha]}{m}.\]
        Тогда $q \mid m$, и, следовательно, $q \leqslant m$.
        \[-\frac{1}{q \tau} \leqslant -\frac{1}{(n+1) m} \leqslant \alpha - \frac{a}{q} \leqslant 0.\]
    \end{proof}

    \begin{remark}
        Такая дробь $\frac{a}{q}$ называется \emph{рациональным приближением $\alpha$ порядка $\tau$}.
    \end{remark}

    \begin{remark}
        Если $\frac{a}{q}$ --- рациональное приближение порядка $\tau$, то $\alpha = \frac{a}{q} + \frac{\theta}{q \tau}$, где $|\theta| \leqslant 1$.
    \end{remark}

    \begin{remark}
        Пусть
        \[E(a, q) := \left(\frac{a}{q} - \frac{1}{q \tau}; \frac{a}{q} + \frac{1}{q \tau}\right).\]
        Тогда $\frac{a}{q}$ будет приближением порядка $\tau$ тогда и только тогда, когда $\alpha \in E(a, q)$.
    \end{remark}

    Теперь пусть $\tau = 6 P^2 = 6N$. Берём $Q \in (1; \frac{\tau}{2})$ (конкретное значение будет позже). Отрезок $[0; 1]$ покрыт объединением
    \[\bigcup_{1 \leqslant q \leqslant \tau} \bigcup_{\substack{a = 0\\ (a, q) = 1}}^q E(a, q).\]
    Пусть $E_1$ --- объединение интервалов $E(a, q)$ с $q \leqslant Q$, а
    \[E_2 := \left(\frac{-1}{\tau}; 1 - \frac{1}{\tau}\right] \setminus E_1\]
    --- всё остальное. Полуинтервал $(-1/\tau; 1-1/\tau]$ был выбран, так как всякий интервал в нём лежит. Действительно, так как $q \leqslant Q \leqslant \tau/2$,
    \begin{gather*}
        \frac{a}{q} - \frac{1}{q\tau} \geqslant \frac{0}{1} - \frac{1}{1 \cdot \tau} = -\frac{1}{\tau}\\
        \frac{a}{q} + \frac{1}{q\tau} \leqslant \frac{q-1}{q} + \frac{1}{q\tau} = 1 - \frac{\tau - 1}{q\tau} \leqslant 1 - \frac{1}{\tau} + \frac{\tau}{\tau^2} - \frac{2\tau - 2}{\tau^2} = 1 - \frac{1}{\tau} - \frac{\tau - 2}{\tau^2} < 1 - \frac{1}{\tau}.
    \end{gather*}

    \begin{theorem}
        $E_1$ состоит из непересекающихся интервалов.
    \end{theorem}

    \begin{proof}
        Предположим противное. Пусть есть точки $\frac{a_1}{q_1}$ и $\frac{a_2}{q_2}$, отрезки которых перекрываются. Следовательно
        \begin{gather*}
            0 < \frac{a_2}{q_2} - \frac{a_1}{q_1} < \frac{1}{q_1 \tau} + \frac{1}{q_2 \tau}\\
            0 < \frac{a_2 q_1 - a_1 q_2}{q_1 q_2} < \frac{q_1 + q_2}{q_1 q_2 \tau}\\
            \frac{1}{q_1 q_2} < \frac{q_1 + q_2}{q_1 q_2 \tau}\\
            \intertext{(так как $a_2 q_1 - a_1 q_2$ --- целое, неотрицательное и неравное $0$, так как $\GCD(a_1, q_1) = \GCD(a_2, q_2) = 1$)}
            \tau < q_1 + q_1 \leqslant 2Q
        \end{gather*}
    \end{proof}

    \[I(N) = \int_{E_1} + \int_{E_2} = I_1(N) + I_2(N)\]
    --- \emph{интегралы по большим и малым дугам}.

    Тригонометрические суммы:
    \[S_1(\alpha) := \sum_{x=0}^P e^{2 \pi i \alpha x}.\]
    Если $\alpha$ целое, то $S_1 = P+1$, иначе
    \[S_1(\alpha) = \frac{e^{2 \pi i \alpha (P+1)} - 1}{e^{2 \pi i \alpha} - 1}.\]
    Таким образом
    \[|S_1(\alpha)| \leqslant \frac{2}{|e^{\pi i \alpha}(e^{\pi i \alpha} - e^{-\pi i \alpha})|} = \frac{1}{\sin(\pi \alpha)} = \frac{1}{\sin(\pi \|\alpha\|)} \leqslant \frac{1}{2 \|\alpha\|},\]
    где
    \[\|\alpha\| := \min_{n \in \ZZ}(|\alpha - n|)\]
    --- расстояние от $\alpha$ до ближайшего целого, так как для $t \in [0; 1/2]$ верно $\sin(\pi t) \geqslant 2t$. Таким образом
    \[|S_1(\alpha)| \leqslant \min\left(P+1, \frac{1}{2\|\alpha\|}\right)\]

    Теперь давайте оценим $I_2(N)$. Квадратичная сумма:
    \[S_2(\alpha, \beta) = \sum_{0 \leqslant x \leqslant P} e^{2\pi i (\alpha x^2 + \beta x)}.\]
    Тогда
    \[
        |S_2|^2
        = \overline{S}_2 \cdot S_2
        = \sum_{0 \leqslant x \leqslant P} e^{-2\pi i (\alpha x^2 + \beta x)} \sum_{0 \leqslant y \leqslant P} e^{2\pi i (\alpha y^2 + \beta y)}
        = \sum_{0 \leqslant x, y \leqslant P} e^{2\pi i (\alpha (y^2 - x^2) + \beta (y - x))}
    \]
    Фиксируя $x$ и делая замену $h := y - x \in [-x; P - x]$, получаем, что
    \[
        |S_2|^2
        = \sum_{0 \leqslant x \leqslant P} \sum_{-x \leqslant h \leqslant P-x} e^{2\pi i (\alpha h^2 + \beta h)} \cdot e^{2\pi i \cdot 2\alpha hx}
    \]
    Пусть $x_1 := \max(0, -h)$, $y_1 := \min(P, P-h)$. Тогда
    \begin{align*}
        |S_2|^2
        &= \sum_{|h| \leqslant P} e^{2\pi i (\alpha h^2 + \beta h)} \sum_{x_1 \leqslant x \leqslant y_1} e^{2\pi i \cdot 2\alpha hx}\\
        &\leqslant \sum_{|h| \leqslant P} \min\left(P+1, \frac{1}{2 \|2\alpha h\|}\right)\\
        \intertext{(так как $|e^{2\pi i (\alpha h^2 + \beta h)}| = 1$, а внутрення сумма содержит не более $P+1$ члена с модулем 1, поэтому $\leqslant P+1$, а также по аналогии с $S_1$ не более $1/(2\|2\alpha h\|)$)}
        &\leqslant P+1 + 2 \sum_{1 \leqslant h \leqslant P} \min\left(P+1, \frac{1}{2 \|2\alpha h\|}\right)\\
        &\leqslant P+1 + 2 \sum_{1 \leqslant h \leqslant P} \min\left(P+1, \frac{1}{\|2\alpha h\|}\right)\\
        \intertext{(так как больше --- не меньше)}
        &\leqslant P+1 + 2 \sum_{1 \leqslant h \leqslant 2P} \min\left(P+1, \frac{1}{\|\alpha h\|}\right)\\
        \intertext{(вообще произошла замена $h := 2h$, и поэтому стоило бы писать, что новое $h$ чётно, но мы не будем, так как это добавит нечётные члены, которые не уменьшат сумму)}
    \end{align*}

    \begin{definition}
        Знак Виноградова --- если $|A| \leqslant c \cdot B$, то
        \[A \ll B.\]
        Если $c = c(\alpha, \beta, \dots)$, то
        \[A \ll_{\alpha, \beta, \dots} B.\]
    \end{definition}

    \begin{lemma}[неравенство Гёльдера]
        Пусть даны $a_m \geqslant 0$. Тогда
        \[\left(\sum_{m=1}^M a_m\right)^k \leqslant M^{k-1} \sum_{m=1}^M a_m^k.\]
    \end{lemma}

    \begin{corollary}
        Если $M, k = \const$, то
        \[\left(\sum_{m=1}^M a_m\right)^k \ll \sum_{m=1}^M a_m^k.\]
    \end{corollary}

    Кубическая сумма:
    \[S_3(\alpha) = \sum_{0 \leqslant x \leqslant P} e^{2\pi i \alpha x^3}.\]
    Тогда по аналогии с $S_2$
    \begin{align*}
        |S_3|^2
        &= \sum_{0 \leqslant x \leqslant P}\; \sum_{-x \leqslant h_1 \leqslant P-x} e^{2\pi i \alpha ((x+h_1)^3 - x^3)}&
        &= \sum_{|h_1| \leqslant P} e^{2\pi i \alpha h_1^3} \sum_{x_1 \leqslant x \leqslant y_1} e^{2\pi i \cdot 3\alpha h_1(x^2 + xh_1)}\\
        &\leqslant \sum_{|h_1| \leqslant P} \left|\sum_{x_1 \leqslant x \leqslant y_1} e^{2\pi i \cdot 3\alpha h_1(x^2 + xh_1)} \right|&
        &\leqslant P+1 + 2\sum_{1 \leqslant |h_1| \leqslant P} \left|\sum_{x_1 \leqslant x \leqslant y_1} e^{2\pi i \cdot 3\alpha h_1(x^2 + xh_1)} \right|
    \end{align*}
    Таким образом
    \begin{align*}
        |S_3|^4
        &\ll P^2 + \left(\sum_{1 \leqslant |h_1| \leqslant P} \left|\sum_{x_1 \leqslant x \leqslant y_1} e^{2\pi i \cdot 3\alpha h_1(x^2 + xh_1)} \right|\right)^2\\
        \intertext{(так как банально представили как $(|S_3|^2)^2$ и использовали неравенство Гёльдера для $k=2$, $M=3$: $(A + B)^2 \ll A^2 + B^2$)}
        &\ll P^2 + P \sum_{1 \leqslant |h_1| \leqslant P} \left|\sum_{x} e^{2\pi i \cdot 3\alpha h_1(x^2 + xh_1)}\right|^2\\
        \intertext{(так как ещё банальнее использовали неравенство Гёльдера для $k=2$, $M \sim P$, где $a_i$ --- это модули внутренних сумм)}
        &\ll P^2 + P \sum_{1 \leqslant |h_1| \leqslant P} \left|S_2(3 \alpha h_1, 3 \alpha h_1^2)\right|^2\\
        &\ll P^2 + P \sum_{1 \leqslant |h_1| \leqslant P} \left(P + \sum_{1 \leqslant h_2 \leqslant 2P} \min\left(P, \frac{1}{\|3 \alpha h_1 h_2\|}\right)\right)\\
        &\ll P^3 + P \sum_{1 \leqslant |h_1| \leqslant P} \sum_{1 \leqslant h_2 \leqslant 2P} \min\left(P, \frac{1}{\|3 \alpha h_1 h_2\|}\right)\\
        &\ll P^3 + 2P \sum_{1 \leqslant h_1 \leqslant P} \sum_{1 \leqslant h_2 \leqslant 2P} \min\left(P, \frac{1}{\|3 \alpha h_1 h_2\|}\right)\\
        &\ll P^3 + P \sum_{1 \leqslant h_1 \leqslant 3P} \sum_{1 \leqslant h_2 \leqslant 2P} \min\left(P, \frac{1}{\|\alpha h_1 h_2\|}\right)\\
        &\ll P^3 + P \sum_{1 \leqslant n \leqslant 6P^2} \tau(n) \min\left(P, \frac{1}{\|\alpha n\|}\right)\\
        \intertext{(где $n = h_1 h_2 \in [1; 6P^2]$, а $\tau(n)$ --- количество делителей $n$, а значит и оценка сверху на количество пар $(h_1; h_2)$ с произведением $n$)}
        &\ll P^3 + P^{1+\varepsilon} \sum_{1 \leqslant n \leqslant 6P^2} \min\left(P, \frac{1}{\|\alpha n\|}\right)
        \intertext{(так как $\forall \varepsilon > 0\; \exists c = c(\varepsilon)\colon \forall n \in \NN \quad \tau(n) \leqslant c(\varepsilon) n^\varepsilon \ll n^\varepsilon$).}
    \end{align*}\goodbreak

    \begin{theorem}
        Пусть $m \in \ZZ$, $T \geqslant 1$, $\alpha = \frac{a}{q} + \frac{\theta}{q^2}$, $|\theta| \leqslant 1$, $\GCD(a, q) = 1$, $q \geqslant 6$,
        \[W = \sum_{m - \frac{q}{2} < n \leqslant m + \frac{q}{2}} \min\left(T, \frac{1}{\|\alpha n\|}\right).\]
        Тогда $W \leqslant 4T + 2q \ln(q)$.
    \end{theorem}

    \begin{proof}
        Пусть $n = m + x$, $-\frac{q}{2} < x \leqslant \frac{q}{2}$. Тогда
        \begin{multline*}
            \alpha n
            = \alpha (m + x)
            = \alpha x + \alpha m
            = \left(\frac{a}{q} + \frac{\theta}{q^2}\right)x + \alpha m\\
            = \frac{ax}{q} + \alpha m + \frac{\theta x}{q^2}
            = \frac{ax + \alpha qm}{q} + \frac{\theta x}{q^2}
            = \frac{ax+b+\theta_1}{q} + \frac{\theta x}{q^2}
            = \frac{ax+b}{q} + r_x,
        \end{multline*}
        где $b$~--- ближайшее целое к $\alpha qm$, тогда $\alpha qm = b + \theta_1$, где $|\theta_1| \leqslant \frac{1}{2}$. При этом
        \[r_x = \frac{\theta_1}{q} + \frac{\theta x}{q^2}.\]
        Вспоминая, что $|\theta_1| \leqslant \frac{1}{2}$, $|\theta| \leqslant 1$, $|x| \leqslant \frac{q}{2}$, имеем, что
        \[
            |r_x|
            \leqslant \left|\frac{\theta_1}{q}\right| + \left|\frac{\theta x}{q^2}\right|
            \leqslant \frac{1}{2q} + \frac{q/2}{q^2}
            = \frac{1}{q}.
        \]
        
        Определим $y$ как остаток $ax + b$ взятый с полуинтервала $(-\frac{q}{2}; \frac{q}{2}]$. Заметим, что так как $x$ пробегает все остатки по модулю $q$ единожды, а $\GCD(a, q) = 1$, то так же пробегает все остатки и $ax + b \equiv y$, быть может, в другом порядке. Также мы имеем
        \[\|\alpha n\| = \left\|\frac{y}{q} + \rho_y\right\|, \qquad \text{ где } \rho_y := r_x.\]
        Поскольку $|\rho_y| = |r_x| \leqslant 1/q$, то есть ровно не более одного остатка $y$ по модулю $q$, что
        \[\left|\frac{y}{q} + \rho_y\right| \geqslant \frac{1}{2}.\]
        Таким образом оценим в сумме значение для этого остатка, а также для остальных остатков, как $T$ сверху. Для остальных $y$ будет верно, что
        \[\left|\frac{y}{q} + \rho_y\right| \leqslant \frac{1}{2},\]
        т.е. $0$ --- ближайшее значение к $\frac{y}{q} + \rho_y$, а значит
        \[\|\alpha n\| = \left\|\frac{y}{q} + \rho_y\right\| = \left|\frac{y}{q} + \rho_y\right| \geqslant \left|\frac{y}{q}\right| - |\rho_y| \geqslant \frac{|y|}{q} - \frac{1}{q} = \frac{|y| - 1}{q}.\]
        Поскольку $|y| - 1$ принимает неприятные значения при $y \in \{0; 1; -1\}$, то оценим и соответствующие им члены суммы тоже как $T$ сверху; остальные члены оценим сверху как
        \[\leqslant \frac{1}{\|\alpha n\|}\frac{q}{|y| - 1}.\]
        Таким образом получаем, что
        \begin{align*}
            W
            &\leqslant 4T + \sum_{\substack{y \in (-\frac{q}{2}; \frac{q}{2}]\\ y \neq 0, 1, -1, \pm \lfloor \frac{q}{2} \rfloor}} \frac{q}{|y| - 1}\\
            &\leqslant 4T + 2q\sum_{y=2}^{\lfloor \frac{q}{2} \rfloor} \frac{1}{y - 1}\\
            &= 4T + 2q\sum_{y=1}^{\lfloor \frac{q}{2} \rfloor - 1} \frac{1}{y}\\
            &\leqslant 4T + 2q\left(1 + \ln\left(\left\lfloor \frac{q}{2} \right\rfloor - 1\right)\right)\\
            &\leqslant 4T + 2q\ln(q) + 2q(1-\ln(2))\\
        \end{align*}
        [получилось хуже, мы не понимаем почему, но нам и такого хватит].
    \end{proof}

    Так мы получаем
    \begin{align*}
        |S_3|^4
        &\ll P^3 + P^{1+\varepsilon} \left(\frac{6P^2}{q} + 1\right) (4P + 2q \ln(q) + 2q(1-\ln(2))\\
        \intertext{(так как мы разбили отрезок $[1; 6P^2]$ на отрезки длины $q$ и применили теорему к каждому из них)}
        &\ll P^3 + P^{1+\varepsilon} \left(\frac{P^3}{q} + P^2  \ln(q) + q \ln(q)\right)\\
        &\ll P^{1+\varepsilon} \left(\frac{P^3}{4} + P^2 \ln(q) + q\ln(q)\right)\\
        &= P^{4+\varepsilon} \left(\frac{1}{4} + \frac{\ln(q)}{P} + \frac{q \ln(q)}{P^2}\right).
    \end{align*}
\end{document}