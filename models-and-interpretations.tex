\documentclass[12pt,a4paper]{article}
\usepackage{math-text}
% \usepackage{todonotes}
% \usepackage{multicol}
% \usepackage{float}

\newcommand{\ZF}{\mathrm{ZF}}
\newcommand{\ZFC}{\mathrm{ZFC}}
\newcommand{\PA}{\mathrm{PA}}
\newcommand{\AC}{\mathrm{AC}}
\newcommand{\CH}{\mathrm{CH}}
\newcommand{\Th}{\mathrm{Th}}
\DeclareMathOperator{\Con}{Con}

\title{Модели и интерпретации.}
\author{Лев Дмитриевич Беклемишев}

\begin{document}
    \maketitle

    Интерпретация $\sim$ перевод с одного языка на другой. Язык логики предикатов (1 порядка).
    \begin{example}\ 
        \begin{enumerate}
            \item Геометрия Лобачевского в Евклидовой: Бельтрами, Клейн, Пуанкаре.
            \item \CC в \RR. (Эйлер?)
            \item Евклидова геометрия в $\RR$. (Декарт?)
            \item Координатизация: числовая система в элементарной геометрии. (Фон Штаудт, Гильберт, М. Хау???)
            \item \RR в $V$. (Дедекинд)
            \item Любая непротиворечивая теория в счётном языке в $(\NN; {+}, {\cdot})$. (Гёдель, Гильберт/Бернайс)
            \item $\ZFC$ в $\PA + \Con_{\ZFC}$ ($\PA$ --- арифметика Пеано).
            \item $\ZF + \AC + \CH$ в $\ZF$ ($\AC$ --- аксиома выбора, $\CH$ --- континуум гипотеза). (Гёдель)
            \item $\ZF + \AC + \neg \CH$ в $\ZF$. (Коэн; Скотт, Соловьёв)
        \end{enumerate}
    \end{example}

    \begin{definition}
        $U \triangleright V$ --- ``$U$ интерпретируется в $V$''.
    \end{definition}

    \begin{lemma}\ 
        \begin{enumerate}
            \item Если $U \triangleright V$, то из непротиворечивости $U$ следует непротиворечивость $V$.
            \item Предыдущее устанавливается (синтаксическими) следствиями и формулами в $\PA$:
                \[\PA \vdash \Con_{U} \rightarrow \Con_{V}.\]
            \item Если $U$ разрешима (алгоритмически), то разрешима и $T$.

                Например. $Q$ --- слабая арифметическая теория --- алгоритмически неразрешима. При этом
                \[\Th(\text{групп}) \triangleright Q,\]
                а следовательно, теория групп алгоритмически неразрешима.
        \end{enumerate}
    \end{lemma}

    \begin{definition}
        Модель теории множеств: $(\{\text{множества}\}; {\in})$. Аксиомы:
        \begin{enumerate}
            \item $x = y \leftrightarrow \forall z\; (z \in x \leftrightarrow z \in y)$,
            \item $\forall x, y\; \exists z\; \forall u\; (u \in z \leftrightarrow u = x \vee u = y)$,
            \item аксиомы объединения и степени,
            \item аксиомы выделения,
            \item аксиома выбора,
            \item аксиома бесконечности.
        \end{enumerate}
    \end{definition}

    \begin{definition}
        \emph{Язык формальной арифметики} --- $(0; s, {+}, {\cdot}; {=})$ с аксиомами:
        \begin{enumerate}
            \item $\forall x\; \neg s(x) = 0$
            \item $\forall x, y\; x = y \leftrightarrow s(x) = s(y)$
            \item $\forall x\; x \neq 0 \rightarrow \exists y\; s(y) = x$
            \item $x + 0 = x$
            \item $x + s(y) = s(x + y)$
            \item $x \cdot 0 = 0$
            \item $x \cdot s(y) = x \cdot y + y$
            \item (схема аксиом индукции) $\forall \Phi(x)\; (\Phi(0) \wedge \forall x\; (\Phi(x) \rightarrow \Phi(s(x)))) \rightarrow \forall x\; \Phi(x)$.
        \end{enumerate}
    \end{definition}
\end{document}