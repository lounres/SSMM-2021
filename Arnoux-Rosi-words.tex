\documentclass[12pt,a4paper]{article}
\usepackage{math-text}
% \usepackage{todonotes}
% \usepackage{multicol}
% \usepackage{float}

\title{Слова Арну-Рози.}
\author{Иван Алексеевич Дынников}

\begin{document}
    \maketitle

    G. Hedlund, M. Morse.

    \begin{definition}
        Будем рассматривать слова над конечным алфавитом $A$, бесконечные в обе стороны. Обозначим для слова
        \[w = \dots a_{-3} a_{-2} a_{-1} a_0 a_1 a_2 a_3 \dots\]
        функцию
        \[f(n) := \text{``число различных факторов (подслов) длины $n$''}.\]
    \end{definition}

    \begin{theorem}
        Если для некоторого $n$
        \[f(n+1) = f(n),\]
        то $w$ периодично.
    \end{theorem}

    \begin{definition}
        Пусть $G_n(w)$ --- граф, где вершины --- подслова длины $n$, рёбро $u \to v$ проводим тогда и только тогда, когда $ub = av$ --- подслово $w$, где $a, b \in A$.
    \end{definition}

    \begin{definition}
        $L_n(w)$ --- все подслова $w$ длины $n$. $L(w) := \bigcup_{n} L_n(w)$ --- все конченые подслова $w$.
    \end{definition}

    \begin{definition}
        $u \in L_n$ называется
        \begin{enumerate}
            \item \emph{смещённым вправо}, если из $u$ есть стрелка (т.е. расширимо справа до подслова $w$),
            \item \emph{смещённым влево}, если в $u$ есть стрелка (т.е. расширимо слева до подслова $w$).
        \end{enumerate}
    \end{definition}

    \begin{definition}
        $w$ называется \emph{возвратным} (\emph{recurrent}), если всякое $u \in L(w)$ встречается в $w$ бесконечное число раз (т.е. всякое ребро $G_n(w)$ посещаемо бесконечное число раз).
    \end{definition}

    \begin{remark}
        Если $w$ возвратно, то $G_n(w)$ сильно связно.
    \end{remark}

    \begin{theorem}
        В случае $f(n) = n+1$ имеем, что $|A| = 2$. В случае возвратного $w$ и $f(n) = n+1$ имеем, что $G_n(w)$ имеет либо две вершины степени 3, остальные 2, либо одна вершина степени 4, остальные 2. В первом случае $f(n+1) = n+2$, а во втором --- $f(n+1) = n+3$.
    \end{theorem}

    Рассмотрим слова $w$, что
    \begin{enumerate}
        \item $f(n) = 2n+1$,
        \item для всякого $n$ есть ровно одно специальное справа и одно специальное слева подслово в $L_n(w)$,
        \item $w$ возвратно.
    \end{enumerate}
\end{document}