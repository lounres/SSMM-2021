\documentclass[12pt,a4paper]{article}
\usepackage{math-text}
\usepackage{todonotes}
% \usepackage{multicol}
% \usepackage{float}

\title{$\wp$-функция Вейерштрасса, ряды Эйзенштейна и модулярные функции.}
\author{Виктор Алексеевич Клепцын}

\DeclareMathOperator{\sign}{sign}

\begin{document}
    \maketitle

    \begin{definition}
        \emph{Элиптическая кривая} --- $E_\Gamma := \CC / \Gamma$, где
        \[\Gamma := \{a z_1 + b z_2 \mid a, b \in \ZZ \wedge z_1/ z_2 \notin \RR\}.\]
    \end{definition}

    \begin{remark}
        Это же банально тор.
    \end{remark}

    \begin{definition}
        \emph{Элиптическая кривая} --- $\{(x; y) \mid y^2 = P(x)\}$, где $P$ свободен от кратных корней.
    \end{definition}

    \begin{definition}
        $f$ называется \emph{голоморфной} в $D \subseteq \CC$, если $f$ комплексно дифференцируемо в точке $D$.
    \end{definition}

    \begin{lemma}[условие Коши-Римана]
        Пусть $f$ дифференцируема в точке $z_0 \in D \subseteq \RR^2$ ($= \CC$). Тогда если $f$ комплексно дифференцируема в $z_0$, то $\partial_x f(z_0) = \partial_y f(z_0) = f'(z_0)$.d
    \end{lemma}

    \begin{lemma}
        Голоморфная функция на области восстанавливается по значениям на границе.
    \end{lemma}

    \begin{lemma}
        Если $f$ голоморфна в $D$, то она бесконечно комплексно дифференцируема и совпадает со своим рядом Тейлора (для всякой внутренней точки в $D$).
    \end{lemma}

    \begin{lemma}
        Голоморфная на области функция имеет экстремумы на области на границе этой области.
    \end{lemma}
    
    \begin{remark}
        Рассмотрим голоморфные функции на элиптической кривой, они же голоморфные функции, периодические по двум неколлинеарным векторам. В таком случае понятно, что она константна.
    \end{remark}

    Будем рассматривать функции не в $\CC$, а в $\CC \sqcup \{\infty\} = \CC P^1$.

    \begin{definition}
        \emph{Полюс порядка $n$} у функции $f$ --- точка $z_0$, что $f(z) = \sum_{k=-n}^\infty c_k (z - z_0)^n$ и $c_{-k} \neq 0$, т.е. $1/f = (z - z_0)^n + \dots$. 
    \end{definition}

    \begin{definition}
        Функция $f$ \emph{мероморфна}, если для всякой точки верно, что либо в её окрестности $f$ голоморфна, либо она является полюсом.
    \end{definition}

    \begin{definition}
        \emph{Элептическая функция} --- мероморфная функция на элиптической кривой.
    \end{definition}

    \begin{definition}
        Пусть дана непрерывная функция $f: S^1 \to S^1$. Тогда \emph{порядком $f$} называется количество оборотов $f$ при прохождении единожды по окружности.
    \end{definition}

    \begin{definition}
        Пусть дана непрерывная дифференцируемая функция $f: S^1 \to S^1$. Тогда для всякой точки $p \in S^1$, 
        если $\{q_1; \dots; q_n\} = f^{-1}(p)$ и $f'(q_i) \neq 0$, то величина
        \[\sum_{i=1}^n \sign(f'(q_i))\]
        является (целой) величиной, независящей от $p$. Она называется \emph{порядком $f$}.
    \end{definition}

    \begin{definition}
        \todo[inline]{Дописать.}
        \[\wp(z) := \frac{1}{z^2} + \sum_{\gamma \in E} \frac{1}{(z-\gamma)^2} - \frac{1}{\gamma^2}.\]
    \end{definition}

    \begin{theorem}
        \[\wp'^2 - 4\wp^3 - A\wp^2 - g\wp - d = 0.\]
    \end{theorem}

    \begin{proof}
        Поскольку функция выше мезоморфна и не имеет особенностей, то голоморфна, а значит константна.
    \end{proof}

    \begin{corollary}
        Отображение
        \[f: \CC / \Gamma \to \CC^2, z \to (\wp(z), \wp'(z))\]
        Переводит тор в кривую $y^2 = 4x^3 + g x + d$.
    \end{corollary}

    \begin{definition}
        \emph{Ряд Эйзенштейна} ---
        \[G_k(\Gamma) := \sum_{\Gamma \setminus \{0\}} \frac{1}{\gamma^k}.\]
    \end{definition}

    \begin{remark}
        Заметим, что
        \[
            \frac{1}{\gamma - z}
            = \frac{1}{\gamma} \frac{1}{1 - \frac{z}{\gamma}}
            = \frac{1}{\gamma} (1 + \frac{z}{\gamma} + \frac{z^2}{\gamma^2} + \dots),
        \]
        следовательно
        \[\frac{1}{(z - \gamma)^2} = \left(\frac{1}{\gamma - z}\right)' = \frac{1}{\gamma^2} + \frac{2z}{\gamma^3} + \frac{3z^2}{\gamma^4} + \dots.\]
        Таким образом
        \begin{align*}
            \wp(z)
            &= \frac{1}{z^2} + 2z \sum \frac{1}{\gamma^3} + 3z^2 \sum \frac{1}{\gamma^4} + 4z^3 \sum \frac{1}{\gamma^5} + \dots\\
            &=: \frac{1}{z^2} + 3 G_4(\Gamma) z^2 + 5 G_6(\Gamma) z^4 + 7 G_8(\Gamma) z^6 + \dots\\
            &= \frac{1}{z^2}(1 + 3 G_4(\Gamma) z^4 + 5 G_6(\Gamma) z^6 + 7 G_8(\Gamma) z^8 + \dots)
        \end{align*}
        А значит
        \begin{align*}
            \wp'(z)
            &= -\frac{2}{z^3} + 6 G_4(\Gamma) z + 20 G_6(\Gamma) z^3 + 42 G_8(\Gamma) z^8 + \dots\\
            &= -\frac{2}{z^3}(1 - 3 G_4(\Gamma) z^4 + 10 G_6(\Gamma) z^6 + 21 G_8(\Gamma) z^8 + \dots)
        \end{align*}
        
        Смотря на первые члены рядов, получаем, что $g = 60 G_4(\Gamma)$ и $d = 140 G_6(\Gamma)$, т.е.
        \[\wp'^2 - 4\wp^3 - A\wp^2 - 60 G_4(\Gamma) \wp - 140 G_6(\Gamma) = 0.\]
    \end{remark}

    \subsection*{Пространство решёток (и модулярная кривая)}
\end{document}