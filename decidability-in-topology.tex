\documentclass[12pt,a4paper]{article}
\usepackage{math-text}
% \usepackage{todonotes}
% \usepackage{multicol}
% \usepackage{float}

\title{Гомологические сферы и алгоритмическая неразрешимость в топологии.}
\author{Александр Александрович Гайфуллин}

\begin{document}
    \maketitle

    \begin{definition}
        Пусть дана поверхность $S$ и на ней некоторая точка $p \in S$. Рассмотрим все петли в $S$ с началом (и концом) в $p$ с точностью до деформации (изотопии) (с сохранением начала-конца). Используя конкатенацию петель как операцию сложения получаем группу, называемую \emph{фундаментальную группу}.
    \end{definition}

    \begin{definition}
        Будем рассматривать формальные комбинации петель. При этом две комбинации эквивалентны, если есть ориентируемая поверхность, ``соединяющая'' эти комбинации (т.е. её граница состоит из совокупности комбинаций, но первая группа ориентирована в одну сторону, а вторая --- в другую), вкладываемая в исходную поверхность так, что соответствующие петли переходят в соответствующие с правильной ориентацией. Получаем элементы \emph{группы гомологий}.

        Сложение в группе --- формальная сумма линейных комбинаций (т.е. $(2\alpha + \beta) + (\gamma - \alpha + \beta) = \alpha + 2\beta + \gamma$).
    \end{definition}

    \begin{theorem}
        Если поверхность линейно связна, то группа гомологий --- абелизация фундаментальной группы.
    \end{theorem}
\end{document}