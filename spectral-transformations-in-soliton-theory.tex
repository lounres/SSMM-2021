\documentclass[12pt,a4paper]{article}
\usepackage{math-text}
\usepackage{todonotes}
% \usepackage{multicol}
% \usepackage{float}

\DeclareMathOperator{\GL}{GL}

\title{Спектральные преобразования в теории солитонов.}
\author{Игорь Моисеевич Кричевер}

\begin{document}
    \maketitle

    Мы будем разговаривать о теории систем солитонов.

    \begin{example}\ 
        \begin{enumerate}
            \item \[\frac{\partial u}{\partial t} + \frac{3}{2} u \frac{\partial u}{\partial x} - \frac{1}{4} \frac{\partial^3 u}{\partial x^3} = 0\]
                описывает волны на воде.
            \item \href{https://ru.wikipedia.org/wiki/Цепочка_Тоды}{Ряды Тоды} (\href{https://en.wikipedia.org/wiki/Toda_lattice}{Toda lattice}): \[\frac{1}{4} \ddot{x}_n = e^{2(x_n - x_{n+1})} - e^{2(x_{n-1} - x_n)}.\]
            \item Система CM: \[\ddot{x}_n = \sum_{m=1}^N \frac{4}{(x_n - x_m)^2}\]
            \item Система ??: \[\sum_{i+j} \frac{2}{u_i^{(m)} - u_j^{(m)}} - \sum_{i+j} \frac{1}{u_i^{(m)} - u_j^{(m+1)}} - \sum_{i+j} \frac{1}{u_i^{(m)} - u_j^{(m-1)}}\]
                Рассмотрим матрицу
                \[
                    M_{i, i} = a_i + t_1 + 2 \mu_i t_2 + 3 \mu_i^2 t_3 + \dots + n \mu_i^{n-1}
                    \qquad
                    M_{i, j} = \frac{1}{\mu_i \cdot \mu_j}
                \]
                Тогда
                \[
                    u = -2 \partial x^2 \ln(\det(M))
                    \qquad
                    u = -2 \partial x^2 \ln(\Theta(Ux + Vy + Wt + z(B))).
                \]
        \end{enumerate}
    \end{example}

    Что общее у всех этих примеров? Интегральные системы $\Leftrightarrow$ условия совместимости переопределённой системы линейных уравнений.

    \begin{example}
        Рассмотрим преобразование $L: C(\RR)^n \to C(\RR)^n, (\psi_k(t))_{k=1}^n \mapsto (L \psi_k(t))_{k=1}^n$, что
        \[L(\psi)_n = c_n \psi_{n+1} + v_n \psi_n + c_{n-1} \psi_{n-1}.\]
        Попытаемся найти собственные вектора $L$. Пусть $L\psi = E \psi$. Также попросим
        \[\partial_t \psi_n = c_{n+1} \psi_{n+1} - c_{n-1} \psi_{n+1}\]
        Тогда
        \begin{align*}
            &\mathop{\phantom{=}} \dot{c}_n \psi_{n+1} + c_n (c_n \psi_{n+2} - c_n \psi_n) + \dot{v}_n \psi_n + v_n (c_n \psi_{n+1} - c_{n-1} \psi_{n-1}) + \dot{c}_{n-1} \psi_{n-1} + c_{n-1} (c_n \psi_n - c_{n-2} \psi_{n-2})\\
            &= c_n(c_{n+1} \psi_{n+2} + v_{n+1} \psi_{n+1} + c_n \psi_n) - c_{n-1} (c_{n-1} \psi_n + v_{n-1} \psi_{n-1} + c_{n-2} \psi_{n-2})
        \end{align*}
        Следовательно
        \[
            \dot{c}_n = c_n (v_{n+1} - v_n)
            \qquad
            \dot{v_n} = 2 (c_n^2 - c_{n-1}^2).
        \]
        Если, например, $c_n = e^{x_n - x_{n-1}}$, а $v_n = \dot{x}_n$;
        \[\ddot{x}_n = Q e^{2(x_n - x_{n+1})} - e^{2(x_n-1 - x_n)}.\]
    \end{example}

    Условие совместимости системы
    \[
        \begin{cases}
            L \psi = E \psi\\
            \partial \psi = A \psi
        \end{cases}
    \]
    --- $\dot{L} = [A, L]$ (уравнение Лакса).

    Фазовое пространство --- пространство $l(z)$ --- пространство в зависимости от ??

    Пусть $c_{n + N} = c_n$, $v_{n + N} = v_n$, $\psi_{n+N} = z \psi_n$ для всех $n$. Тогда
    \[
        L = 
        \begin{pmatrix}
            v_n& c_n&& z c_0\\
            & \ddots& \ddots\\
            && \ddots& \ddots\\
            z^{-1} c_0 &&& \ddots
        \end{pmatrix}
    \]

    Рассмотрим матрицу
    \[
        L(z) =
        \begin{pmatrix}
            l_{1,1}(z)& l_{1, 2}(z)\\
            l_{2,1}(z)& l_{2, 2}(z)\\
        \end{pmatrix},
    \]
    где
    \[l_{i,j}(z) = \sum_{k=0}^n l_{i,j,k} z^k.\]
    Найдём её собственные значения $w$.
    \[\det(L(z) - w I) = w^2 - (l_{1,1}(z) + l_{2,2}(z)) w + (l_{1,1}(z) l_{2,2}(z) - l_{1,2}(z) l_{2,1}(z)) = w^2 + Q_n(z) w + R_{2n}(z) = 0.\]
    Заметим, что можно смотреть на $L$ с точностью до $\GL(z)$, т.е. $(L_{i,j,n})$ имеет форму Жордана, но будем считать, что диагональна. Таким образом
    \[w = \frac{-Q \pm \sqrt{Q^2 - 4R}}{2},\]
    а соответствующий собственный вектор --- $\left(\begin{smallmatrix} l_{1,2} \\ w-l_{1,1} \end{smallmatrix}\right) \sim \left(\begin{smallmatrix}1\\\psi\end{smallmatrix}\right)$.

    Дивизор полюсов $\psi$: $l_{1,2}(\gamma_s) = 0$ ($s \in \{1; \dots; n-1\}$). Для одного $z$ не более $1$ полюса.

    Кривая рода $g = n-1$ и дивизор степени $g+1 = n$.

    \begin{theorem}[Римана-Роха]
        Размерность пространства мероморфных функций с полюсом в $d$ различных точках общего положения.
    \end{theorem}

    $y^2 = P_{2n}(z)$ и $y = (w + \frac{Q}{z})$. Тогда всякая $f$ имеет вид
    \[\frac{r_1(z) y + r_2(z)}{r_3(z)}.\]
    Видимо, потому что
    \[r_3(z) = \prod_{s=1}^d (z - \tilde{\gamma}_s), \quad r_2(\gamma_s) = r_1(\gamma_s) \sqrt{P_{2n}(\gamma_s)}.\]
    
    \[L(z) = \hat{\psi}(z) \hat{w}(z) \hat{\psi}^{-1}(z),\]
    где $\hat{w}$ диагональна.
    \[
        \hat{w} = \begin{pmatrix}
            w^+(z)\\
            & w^-(z)
        \end{pmatrix}
    \]
    Таким образом
    \[
        L(z) = \frac{1}{\psi^- - \psi^+}
        \begin{pmatrix}
            1& 1\\
            \psi^+& \psi^-
        \end{pmatrix}
        \begin{pmatrix}
            w^+\\
            &w^-
        \end{pmatrix}
        \begin{pmatrix}
            \psi^-& 1\\
            -\psi^+& 1
        \end{pmatrix}
    \]

    \[\dot{L} = [A, L],\]
    где
    \[A = \begin{pmatrix}
        a_{1,1}(z)& a_{1,2}(z)\\
        a_{2,1}(z)& a_{2,2}(z)
    \end{pmatrix}\]

    \[\hat{L} = \sum \frac{u_i}{z - z_i} = \frac{L(z)}{\prod (z - z_i)}\]
\end{document}