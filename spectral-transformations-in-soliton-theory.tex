\documentclass[12pt,a4paper]{article}
\usepackage{math-text}
\usepackage{todonotes}
% \usepackage{multicol}
% \usepackage{float}

\title{Спектральные преобразования в теории солитонов.}
\author{Игорь Моисеевич Кричевер}

\DeclareMathOperator{\Quot}{Quot}
\DeclareMathOperator*{\osc}{osc}
\DeclareMathOperator{\sign}{sign}
\DeclareMathOperator{\const}{const}
\DeclareMathOperator{\grad}{grad}
\newcommand{\eqdef}{\mathbin{\stackrel{\mathrm{def}}{=}}}
\newcommand{\True}{\mathrm{True}}
\newcommand{\False}{\mathrm{False}}

\begin{document}
    \maketitle

    Мы будем разговаривать о теории систем солитонов.

    \begin{example}\ 
        \begin{enumerate}
            \item \[\frac{\partial u}{\partial t} + \frac{3}{2} u \frac{\partial u}{\partial x} - \frac{1}{4} \frac{\partial^3 u}{\partial x^3} = 0\]
                описывает волны на воде.
            \item Tola: \[\frac{1}{4} \ddot{x}_n = e^{2(x_n - x_{n+1})} - e^{2(x_{n-1} - x_n)}.\]
            \item Система CM: \[\ddot{x}_n = \sum_{m=1}^N \frac{4}{(x_n - x_m)^2}\]
            \item Система ??: \[\sum_{i+j} \frac{2}{u_i^{(m)} - u_j^{(m)}} - \sum_{i+j} \frac{1}{u_i^{(m)} - u_j^{(m+1)}} - \sum_{i+j} \frac{1}{u_i^{(m)} - u_j^{(m-1)}}\]
                Рассмотрим матрицу
                \[
                    M_{i, i} = a_i + t_1 + 2 \mu_i t_2 + 3 \mu_i^2 t_3 + \dots + n \mu_i^{n-1}
                    \qquad
                    M_{i, j} = \frac{1}{\mu_i \cdot \mu_j}
                \]
                Тогда
                \[
                    u = -2 \partial x^2 \ln(\det(M))
                    \qquad
                    u = -2 \partial x^2 \ln(\Theta(Ux + Vy + Wt + z(B))).
                \]
        \end{enumerate}
    \end{example}

    Что общее у всех этих примеров? Интегральные системы $\Leftrightarrow$ условия совместимости переопределённой системы линейных уравнений.

    \begin{example}
        Рассмотрим преобразование $L: C(\RR)^n \to C(\RR)^n, (\psi_k(t))_{k=1}^n \mapsto (L \psi_k(t))_{k=1}^n$, что
        \[L(\psi)_n = c_n \psi_{n+1} + v_n \psi_n + c_{n-1} \psi_{n-1}.\]
        Попытаемся найти собственные вектора $L$. Пусть $L\psi = E \psi$. Также попросим
        \[\partial_t \psi_n = c_{n+1} \psi_{n+1} - c_{n-1} \psi_{n+1}\]
        Тогда
        \begin{align*}
            &\mathop{\phantom{=}} \dot{c}_n \psi_{n+1} + c_n (c_n \psi_{n+2} - c_n \psi_n) + \dot{v}_n \psi_n + v_n (c_n \psi_{n+1} - c_{n-1} \psi_{n-1}) + \dot{c}_{n-1} \psi_{n-1} + c_{n-1} (c_n \psi_n - c_{n-2} \psi_{n-2})\\
            &= c_n(c_{n+1} \psi_{n+2} + v_{n+1} \psi_{n+1} + c_n \psi_n) - c_{n-1} (c_{n-1} \psi_n + v_{n-1} \psi_{n-1} + c_{n-2} \psi_{n-2})
        \end{align*}
        Следовательно
        \[
            \dot{c}_n = c_n (v_{n+1} - v_n)
            \qquad
            \dot{v_n} = 2 (c_n^2 - c_{n-1}^2).
        \]
        Если, например, $c_n = e^{x_n - x_{n-1}}$, а $v_n = \dot{x}_n$;
        \[\ddot{x}_n = Q e^{2(x_n - x_{n+1})} - e^{2(x_n-1 - x_n)}.\]
    \end{example}

    Условие совместимости системы
    \[
        \begin{cases}
            L \psi = E \psi\\
            \partial \psi = A \psi
        \end{cases}
    \]
    --- $\dot{L} = [A, L]$ (уравнение Лакса).

    Фазовое пространство --- пространство $l(z)$ --- пространство в зависимости от ??

    Пусть $c_n = c_{n + N}$, $v_n = v_{n + N}$. Тогда
    \[
        L = 
        \begin{pmatrix}
            v_n& c_n&& z c_0\\
            & \ddots& \ddots\\
            && \ddots& \ddots\\
            z^{-1} c_0 &&& \ddots
        \end{pmatrix}
    \]
\end{document}