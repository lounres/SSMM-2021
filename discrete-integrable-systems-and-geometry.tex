\documentclass[12pt,a4paper]{article}
\usepackage{math-text}
% \usepackage{todonotes}
% \usepackage{multicol}
% \usepackage{float}

\title{Дискретные интегрируемые системы и геометрия.}
\author{Антон Викторович Джамай}

\begin{document}
    \maketitle

    Что такое динамическая система? Это система, которая меняется с течением ``времени''. В непрерывном случае у нас есть конфигурационное пространство $X$, после чего мы рассматриваем некоторое отображение $[0; 1] \to X$ (или $[0; \RR) \to X$), удовлетворяющее определённому дифференциальному уравнению --- функция движения системы. В дискретном случае мы будем рассматривать $\{1; \dots; n\} \to X$ (или $\NN \to X$), которое будет удовлетворять рекурентному соотношению.

    В непрерывном случае мы можем рассматривать
    \begin{itemize}
        \item интегрируемость,
        \item нелинейность,
        \item обратимость.
    \end{itemize}
    Хотим того же для дискретного случая.

    \begin{example}
        Числа Фиббоначчи:
        \[F_0 = 0, \quad F_1 = 1, \quad F_2 = 1, \quad F_3 = 2, \dots\]
        С одной стороны
        \[F_{n+2} = F_{n+1} + F_n.\]
        С другой стороны
        \[
            \begin{pmatrix}
                F_{n+2}\\
                F_{n+1}
            \end{pmatrix}
            =
            \begin{pmatrix}
                1& 1\\
                1& 0
            \end{pmatrix}
            \begin{pmatrix}
                F_{n+1}\\
                F_n
            \end{pmatrix},
        \]
        что указывает на линейность системы.
    \end{example}

    \begin{definition}
        \emph{Отображение QRT (G.R.W. Qinspil, L.A.G. Roberts, C. Thomas)}. Пусть есть кривая $\Gamma: p(x, y) = 0$, $\deg_x(p) = \deg_y(p) = 2$ (биквадратный многочлен, $(2, 2)$).

        Чтобы иметь алгебраическую замкнутость, будем сидеть в $\CC^2$, а чтобы не терять корней, будем сидеть в $(\CC P^1)^2$.

        Для всякой точки $(x_0, y_0) \in \Gamma \subseteq (\CC P^1)^2$ можно рассмотреть прямую $l: x = x_0$. Эта прямая пересечёт $\Gamma$ в двух точках: $(x_0, y_0)$ и $(x_0, y_0')$. Таким образом мы получаем инволюцию
        \[r_x: \Gamma \to \Gamma, (x_0, y_0) \mapsto (x_0, y_0');\]
        аналогично определяется $r_y$. Обе $r_x$ и $r_y$ являются инволюциями. Таким образом рассматриваемое отображение QRT будет
        \[\eta := r_y \circ r_x, \qquad \eta: (\CC P^1)^2 \to (\CC P^1)^2, (x_0, y_0) \mapsto (\overline{x_0}, \overline{y_0}).\]
    \end{definition}

    \begin{definition}
        Пусть
        \[p(x, y) = \sum_{d_1, d_2 \in \{0; 1; 2\}} a_{d_1, d_2} x^{d_1} y^{d_2}.\]
        Тогда $p(x, y) = 0$ записывается как
        \[
            \begin{pmatrix}
                1& x& x^2
            \end{pmatrix}
            \begin{pmatrix}
                a_{0,0}& a_{0, 1}& a_{0, 2}\\
                a_{1,0}& a_{1, 1}& a_{1, 2}\\
                a_{2,0}& a_{2, 1}& a_{2, 2}
            \end{pmatrix}
            \begin{pmatrix}
                1\\
                y\\
                y^2
            \end{pmatrix}
            = 0
            \qquad \text{ или } \qquad
            \vec{x}^T A \vec{y} = 0.
        \]
    \end{definition}
    
    \begin{remark}
        Пусть $r_y(x_0, y_0) = (x_0, y_1)$. Тогда по теореме Виета
        \[y_0 + y_1 = \frac{-a_1(x_0)}{a_2(x_0)},\]
        где временно $p(x, y) = a_2(x) y^2 + a_1(x) y + a_0$.
    \end{remark}

    \begin{definition}
        Теперь пусть у нас есть кривые $\Gamma_A$ и $\Gamma_B$ с матрицами $A$ и $B$. Для всякой точки $(x_0, y_0)$, не лежащей не пересечении $\Gamma_A \cap \Gamma_B$ (пересечение в общем случае имеет 8 точек), имеет $\vec{x_0}^T A \vec{y_0}$ и $\vec{x_0}^T B \vec{y_0}$, где одно не равно $0$. Следовательно можно найти их однородную комбинацию равную $0$:
        \[\vec{x_0}^T (\lambda_0 A + \lambda_1 B) \vec{y_0} = 0, \qquad [\lambda_0: \lambda_1] = [- \vec{x_0}^T B \vec{y_0}: \vec{x_0}^T A \vec{y_0}].\]
        Следовательно можно рассмотреть преобразование почти всего $(\CC P^1)^2$
        \[
            \eta(x_0, y_0)
            = \eta_{\lambda_0 A + \lambda_1 B}(x_0, y_0)
            = \dots
        \]
    \end{definition}

    \begin{remark}
        Поскольку при $\eta$ мы применяем преобразование $\eta_{\lambda_0 A + \lambda_1 B}$, то точка остаётся на кривой, а значит мы остаёмся на той же кривой, а значит значение $[\lambda_0: \lambda_1]$ не меняется и называется \emph{интегралом}.
    \end{remark}
\end{document}